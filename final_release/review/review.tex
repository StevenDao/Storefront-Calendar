\documentclass[12pt]{article}

% Misc. packages
\usepackage{array,calc,lastpage}
\usepackage{amsmath,amssymb}
\usepackage{graphicx,subfig}

% Page layout
\usepackage[left=20mm, right=20mm, top=30mm, bottom=25mm]{geometry}
\flushbottom

% Headers and footers
\usepackage{fancyhdr}
\lhead{{}}
\chead{{}}
\rhead{{}}
\lfoot{{}}
\cfoot{{}}
\rfoot{{Page \thepage\ of \pageref{LastPage}}}

% Sectioning commands
\setcounter{secnumdepth}{-1}
%\titleformat{\section}{\large\bfseries}{\thesection}{1em}{}

% Paragraph style
\raggedright
\setlength{\parindent}{0pt}
\setlength{\parskip}{8pt}



\begin{document}

\section{Review of Development Process}

The development process used was agile. Throughout the term, the overhead
associated with using the agile process was much greater than the actual
development process. This overhead would have been productive with a project
that is on a greater scale and magnitude, however, since most of the time used
on this project was on the agile process, the set of functionality is just a
subset of the intended design.

Although Test Driven Development (TDD) was encouraged for this project, it was
difficult to pursue due to the technologies we chose. The framework CodeIgniter
doesn't support unit tests and integration testing making it difficult to
integrate TDD into our development cycle. In hindsight, not having testing in
our development cycle introduced a multitude of mini-bugs dealing with form
validation and such. With a test suite, these can be worked out and fixed as
strict tests against the system, but this would have cut down development time
by half and the feature set would be much smaller.

Team development techniques in a university course is not something we would
recommend due to the limited time per week spent with a team. Although the
project was over a span of a few months, the actual time spent was only a fifth
of that due to other courses conflicting with time. With this in mind, it's too
much overhead to follow an agile process completely. The user stories in the
beginning were a good guideline to the future functionality in the form of a
checklist.

\end{document}

